\PassOptionsToPackage{unicode=true}{hyperref} % options for packages loaded elsewhere
\PassOptionsToPackage{hyphens}{url}
%
\documentclass[
]{article}
\usepackage{lmodern}
\usepackage{amssymb,amsmath}
\usepackage{ifxetex,ifluatex}
\ifnum 0\ifxetex 1\fi\ifluatex 1\fi=0 % if pdftex
  \usepackage[T1]{fontenc}
  \usepackage[utf8]{inputenc}
  \usepackage{textcomp} % provides euro and other symbols
\else % if luatex or xelatex
  \usepackage{unicode-math}
  \defaultfontfeatures{Scale=MatchLowercase}
  \defaultfontfeatures[\rmfamily]{Ligatures=TeX,Scale=1}
\fi
% use upquote if available, for straight quotes in verbatim environments
\IfFileExists{upquote.sty}{\usepackage{upquote}}{}
\IfFileExists{microtype.sty}{% use microtype if available
  \usepackage[]{microtype}
  \UseMicrotypeSet[protrusion]{basicmath} % disable protrusion for tt fonts
}{}
\makeatletter
\@ifundefined{KOMAClassName}{% if non-KOMA class
  \IfFileExists{parskip.sty}{%
    \usepackage{parskip}
  }{% else
    \setlength{\parindent}{0pt}
    \setlength{\parskip}{6pt plus 2pt minus 1pt}}
}{% if KOMA class
  \KOMAoptions{parskip=half}}
\makeatother
\usepackage{xcolor}
\IfFileExists{xurl.sty}{\usepackage{xurl}}{} % add URL line breaks if available
\IfFileExists{bookmark.sty}{\usepackage{bookmark}}{\usepackage{hyperref}}
\hypersetup{
  pdftitle={What gets sanitized},
  pdfborder={0 0 0},
  breaklinks=true}
\urlstyle{same}  % don't use monospace font for urls
\usepackage[margin=1in]{geometry}
\usepackage{color}
\usepackage{fancyvrb}
\newcommand{\VerbBar}{|}
\newcommand{\VERB}{\Verb[commandchars=\\\{\}]}
\DefineVerbatimEnvironment{Highlighting}{Verbatim}{commandchars=\\\{\}}
% Add ',fontsize=\small' for more characters per line
\usepackage{framed}
\definecolor{shadecolor}{RGB}{248,248,248}
\newenvironment{Shaded}{\begin{snugshade}}{\end{snugshade}}
\newcommand{\AlertTok}[1]{\textcolor[rgb]{0.94,0.16,0.16}{#1}}
\newcommand{\AnnotationTok}[1]{\textcolor[rgb]{0.56,0.35,0.01}{\textbf{\textit{#1}}}}
\newcommand{\AttributeTok}[1]{\textcolor[rgb]{0.77,0.63,0.00}{#1}}
\newcommand{\BaseNTok}[1]{\textcolor[rgb]{0.00,0.00,0.81}{#1}}
\newcommand{\BuiltInTok}[1]{#1}
\newcommand{\CharTok}[1]{\textcolor[rgb]{0.31,0.60,0.02}{#1}}
\newcommand{\CommentTok}[1]{\textcolor[rgb]{0.56,0.35,0.01}{\textit{#1}}}
\newcommand{\CommentVarTok}[1]{\textcolor[rgb]{0.56,0.35,0.01}{\textbf{\textit{#1}}}}
\newcommand{\ConstantTok}[1]{\textcolor[rgb]{0.00,0.00,0.00}{#1}}
\newcommand{\ControlFlowTok}[1]{\textcolor[rgb]{0.13,0.29,0.53}{\textbf{#1}}}
\newcommand{\DataTypeTok}[1]{\textcolor[rgb]{0.13,0.29,0.53}{#1}}
\newcommand{\DecValTok}[1]{\textcolor[rgb]{0.00,0.00,0.81}{#1}}
\newcommand{\DocumentationTok}[1]{\textcolor[rgb]{0.56,0.35,0.01}{\textbf{\textit{#1}}}}
\newcommand{\ErrorTok}[1]{\textcolor[rgb]{0.64,0.00,0.00}{\textbf{#1}}}
\newcommand{\ExtensionTok}[1]{#1}
\newcommand{\FloatTok}[1]{\textcolor[rgb]{0.00,0.00,0.81}{#1}}
\newcommand{\FunctionTok}[1]{\textcolor[rgb]{0.00,0.00,0.00}{#1}}
\newcommand{\ImportTok}[1]{#1}
\newcommand{\InformationTok}[1]{\textcolor[rgb]{0.56,0.35,0.01}{\textbf{\textit{#1}}}}
\newcommand{\KeywordTok}[1]{\textcolor[rgb]{0.13,0.29,0.53}{\textbf{#1}}}
\newcommand{\NormalTok}[1]{#1}
\newcommand{\OperatorTok}[1]{\textcolor[rgb]{0.81,0.36,0.00}{\textbf{#1}}}
\newcommand{\OtherTok}[1]{\textcolor[rgb]{0.56,0.35,0.01}{#1}}
\newcommand{\PreprocessorTok}[1]{\textcolor[rgb]{0.56,0.35,0.01}{\textit{#1}}}
\newcommand{\RegionMarkerTok}[1]{#1}
\newcommand{\SpecialCharTok}[1]{\textcolor[rgb]{0.00,0.00,0.00}{#1}}
\newcommand{\SpecialStringTok}[1]{\textcolor[rgb]{0.31,0.60,0.02}{#1}}
\newcommand{\StringTok}[1]{\textcolor[rgb]{0.31,0.60,0.02}{#1}}
\newcommand{\VariableTok}[1]{\textcolor[rgb]{0.00,0.00,0.00}{#1}}
\newcommand{\VerbatimStringTok}[1]{\textcolor[rgb]{0.31,0.60,0.02}{#1}}
\newcommand{\WarningTok}[1]{\textcolor[rgb]{0.56,0.35,0.01}{\textbf{\textit{#1}}}}
\usepackage{graphicx,grffile}
\makeatletter
\def\maxwidth{\ifdim\Gin@nat@width>\linewidth\linewidth\else\Gin@nat@width\fi}
\def\maxheight{\ifdim\Gin@nat@height>\textheight\textheight\else\Gin@nat@height\fi}
\makeatother
% Scale images if necessary, so that they will not overflow the page
% margins by default, and it is still possible to overwrite the defaults
% using explicit options in \includegraphics[width, height, ...]{}
\setkeys{Gin}{width=\maxwidth,height=\maxheight,keepaspectratio}
\setlength{\emergencystretch}{3em}  % prevent overfull lines
\providecommand{\tightlist}{%
  \setlength{\itemsep}{0pt}\setlength{\parskip}{0pt}}
\setcounter{secnumdepth}{5}
% Redefines (sub)paragraphs to behave more like sections
\ifx\paragraph\undefined\else
  \let\oldparagraph\paragraph
  \renewcommand{\paragraph}[1]{\oldparagraph{#1}\mbox{}}
\fi
\ifx\subparagraph\undefined\else
  \let\oldsubparagraph\subparagraph
  \renewcommand{\subparagraph}[1]{\oldsubparagraph{#1}\mbox{}}
\fi

% set default figure placement to htbp
\makeatletter
\def\fps@figure{htbp}
\makeatother

\usepackage{threeparttable}
\usepackage{booktabs}
\usepackage{longtable}
\usepackage[utopia]{mathdesign}
\usepackage{array}

\title{What gets sanitized}
\author{}
\date{\vspace{-2.5em}}

\begin{document}
\maketitle

{
\setcounter{tocdepth}{2}
\tableofcontents
}
\clearpage

\hypertarget{notes-are-sanitized}{%
\section{Notes are sanitized}\label{notes-are-sanitized}}

\begin{Shaded}
\begin{Highlighting}[]
\NormalTok{x <-}\StringTok{ }\KeywordTok{ptdata}\NormalTok{() }\OperatorTok\StringTok{ }\KeywordTok{st_new}\NormalTok{(}\DataTypeTok{notes =} \StringTok{"EDA_summary = TRUE"}\NormalTok{) }\OperatorTok\StringTok{ }
\StringTok{  }\KeywordTok{st_make}\NormalTok{(}\DataTypeTok{inspect =} \OtherTok{TRUE}\NormalTok{) }\OperatorTok\StringTok{ }
\StringTok{  }\KeywordTok{get_stable_data}\NormalTok{() }

\NormalTok{x}\OperatorTok{$}\NormalTok{notes}
\end{Highlighting}
\end{Shaded}

\begin{verbatim}
## [1] "EDA\\_summary = TRUE"
\end{verbatim}

\clearpage

\hypertarget{file-names-are-sanitized}{%
\section{File names are sanitized}\label{file-names-are-sanitized}}

\begin{Shaded}
\begin{Highlighting}[]
\NormalTok{x <-}\StringTok{ }\KeywordTok{ptdata}\NormalTok{() }\OperatorTok\StringTok{ }\KeywordTok{st_new}\NormalTok{() }\OperatorTok\StringTok{ }
\StringTok{  }\KeywordTok{st_files}\NormalTok{(}\DataTypeTok{r =} \StringTok{"my_script.R"}\NormalTok{) }\OperatorTok\StringTok{ }
\StringTok{  }\KeywordTok{st_make}\NormalTok{(}\DataTypeTok{inspect =} \OtherTok{TRUE}\NormalTok{) }\OperatorTok\StringTok{ }
\StringTok{  }\KeywordTok{get_stable_data}\NormalTok{() }

\NormalTok{x}\OperatorTok{$}\NormalTok{notes}
\end{Highlighting}
\end{Shaded}

\begin{verbatim}
## [1] "Source code: my\\_script.R"
\end{verbatim}

\hypertarget{column-names-are-sanitized}{%
\section{Column names are sanitized}\label{column-names-are-sanitized}}

\begin{Shaded}
\begin{Highlighting}[]
\NormalTok{out <-}\StringTok{ }
\StringTok{  }\KeywordTok{tibble}\NormalTok{(}\DataTypeTok{a_1 =} \DecValTok{5}\NormalTok{) }\OperatorTok\StringTok{ }
\StringTok{  }\KeywordTok{stable}\NormalTok{(}\DataTypeTok{inspect =} \OtherTok{TRUE}\NormalTok{) }\OperatorTok\StringTok{ }
\StringTok{  }\KeywordTok{get_stable_data}\NormalTok{()}

\NormalTok{out}\OperatorTok{$}\NormalTok{cols_new}
\end{Highlighting}
\end{Shaded}

\begin{verbatim}
## [1] "a_1"
\end{verbatim}

\hypertarget{data-are-sanitized}{%
\section{Data are sanitized}\label{data-are-sanitized}}

\begin{Shaded}
\begin{Highlighting}[]
\NormalTok{out <-}\StringTok{ }
\StringTok{  }\KeywordTok{tibble}\NormalTok{(}\DataTypeTok{a =} \StringTok{"5_2"}\NormalTok{) }\OperatorTok\StringTok{ }
\StringTok{  }\KeywordTok{stable}\NormalTok{(}\DataTypeTok{inspect =} \OtherTok{TRUE}\NormalTok{) }\OperatorTok\StringTok{ }
\StringTok{  }\KeywordTok{get_stable_data}\NormalTok{()}

\NormalTok{out}\OperatorTok{$}\NormalTok{tab}
\end{Highlighting}
\end{Shaded}

\begin{verbatim}
## [1] "5\\_2 \\\\"
\end{verbatim}

\hypertarget{span-titles---sanitized}{%
\section{Span titles - sanitized}\label{span-titles---sanitized}}

\begin{Shaded}
\begin{Highlighting}[]
\NormalTok{out <-}\StringTok{ }
\StringTok{  }\KeywordTok{ptdata}\NormalTok{() }\OperatorTok\StringTok{ }
\StringTok{  }\KeywordTok{stable}\NormalTok{(}\DataTypeTok{inspect =} \OtherTok{TRUE}\NormalTok{, }\DataTypeTok{span =} \KeywordTok{colgroup}\NormalTok{(}\StringTok{"foo_this"}\NormalTok{, WT}\OperatorTok{:}\NormalTok{SCR)) }\OperatorTok\StringTok{ }
\StringTok{  }\KeywordTok{get_stable_data}\NormalTok{()}

\NormalTok{out}\OperatorTok{$}\NormalTok{span_data}
\end{Highlighting}
\end{Shaded}

\begin{verbatim}
## $tex
## [1] "\\multicolumn{4}{c}{} & \\multicolumn{5}{c}{\\textbf{foo\\_this}}\\\\"
## [2] "\\cmidrule(lr){5-9}"                                                  
## 
## $cols
## [1] "STUDY" "DOSE"  "FORM"  "N"     "WT"    "CRCL"  "AGE"   "ALB"   "SCR"  
## 
## $span
## $span$`1`
## # A tibble: 9 x 7
##    coln col   newcol title      level   flg align
##   <int> <chr> <chr>  <chr>      <dbl> <dbl> <chr>
## 1     1 STUDY STUDY  ""             1     1 c    
## 2     2 DOSE  DOSE   ""             1     1 c    
## 3     3 FORM  FORM   ""             1     1 c    
## 4     4 N     N      ""             1     1 c    
## 5     5 WT    WT     "foo_this"     1     2 c    
## 6     6 CRCL  CRCL   "foo_this"     1     2 c    
## 7     7 AGE   AGE    "foo_this"     1     2 c    
## 8     8 ALB   ALB    "foo_this"     1     2 c    
## 9     9 SCR   SCR    "foo_this"     1     2 c
\end{verbatim}

\hypertarget{panel-names}{%
\section{Panel names}\label{panel-names}}

\begin{Shaded}
\begin{Highlighting}[]
\NormalTok{data <-}\StringTok{ }\KeywordTok{tibble}\NormalTok{(}\DataTypeTok{a =} \KeywordTok{c}\NormalTok{(}\StringTok{"a_1"}\NormalTok{, }\StringTok{"a_1"}\NormalTok{, }\StringTok{"a_1"}\NormalTok{, }\StringTok{"a_2"}\NormalTok{, }\StringTok{"a_2"}\NormalTok{), }
               \DataTypeTok{b =}\NormalTok{ letters[}\DecValTok{1}\OperatorTok{:}\DecValTok{5}\NormalTok{])}

\NormalTok{out <-}\StringTok{ }\KeywordTok{stable}\NormalTok{(data, }\DataTypeTok{panel =} \StringTok{"a"}\NormalTok{) }
\NormalTok{out[}\KeywordTok{grepl}\NormalTok{(}\StringTok{"multicolumn"}\NormalTok{, out)]}
\end{Highlighting}
\end{Shaded}

\begin{verbatim}
## [1] "\\multicolumn{1}{l}{\\textbf{a\\_1}}\\\\"        
## [2] "\\hline \\multicolumn{1}{l}{\\textbf{a\\_2}}\\\\"
\end{verbatim}

\end{document}
